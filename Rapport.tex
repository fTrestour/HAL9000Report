\documentclass[a4paper,oneside]{report}

\usepackage[utf8]{inputenc}
\usepackage[T1]{fontenc}
\usepackage[francais]{babel}
\usepackage{setspace}
\usepackage{layout}
\usepackage{amsmath}
\usepackage{amssymb}
\usepackage{mathrsfs}
\usepackage{graphicx}
\usepackage{hyperref}
\usepackage{caption}
\usepackage{subcaption}




\renewcommand{\contentsname}{Sommaire}
\pagestyle{headings}



\title{Rapport projet long\\Équipe HAL9000}
\author{Hamelain Christian, Hasan Pierre-Yves, Gaspart Quentin, Trestour Fabien}


\begin{document}


    \setcounter{tocdepth}{1}


    \maketitle


    \tableofcontents


    \part{Introduction}

        \section[Contexte]{Contexte du projet}

            CentraleSupélec propose à  ses étudiants en deuxième année de participer
            à  des projets longs qui se réalisent en équipe, tout au long de l'année sur un thême 
mêlant plusieurs compétences que l'ingénieur se doit de posséder.

            Dans notre cas, c'est l'élaboration de réseaux neuronaux et les méthodes de Deep Learning 
que 12 éléves ont étudié, répartis en 3 groupes, dont HAL9000.

            Ce rapport présente les avancement de ce dernier groupe sur le sujet tout au long de 
l'année.


        \section{Présentation du sujet}

            Ce projet aborde le sujet du Deep Learning. Cette méthode spécifique de machine learning 
est dérivée du concept de réseau de neurones.

            Les réseaux de neurone sont une modélisation simple du fonctionnement cérébral
            à  l'échelle cellulaire. Cette approche a vu le jour avec les études de McCulloch et Pitts 
dès la fin des années 50. Malgré certains écueils dans les années 70, l'approche 
connexioniste des réseaux de 
neurones a su se développer et devenir un sujet de recherche populaire dans les dernières années, 
notament grâce aux capacités d'adaptabilité et de généralisation de ces réseaux qui en font 
d'excellents 
candidats pour des applications telles que la reconnaissance d'image ou la classification.


        \section[Déroulement]{Déroulement du projet}

            Tout d'abord, il s'agissait pour nous de comprendre le fonctionnement de ces structures et 
commencer à  coder des structures élémentaires afin de comprendre les enjeux du deep 
learning. Nous avons pour 
cela travaillé avec la base de données MNIST de Yann LeCun et en JAVA. Afin de travailler en groupe de 
manière efficace, nous avons aussi utilisé l'outil GIT.

            Par la suite nous nous sommes interessés aux architectures profondes, chaque groupe se 
penchant sur une architecture spécifique. Notre équipe s'est en particulier intéressée aux 
machines de Boltzmann, en 
commençant par les machines de Boltzmann restreintes (RBM), puis la structure de Deep Learning 
associée, les Deep Belief Networks.

            Cette étude a été encadrée par Joanna Tomasik et Arpad Rimmel, enseignants à  
CentraleSupélec, campus de Gif-sur-Yvette.



    \part[Le Perceptron]{Première approche du problème : le Perceptron}


        \chapter{Architecture d'un perceptron}

            \section{Introduction}

                Ici, l'objectif était la reconnaissance des caractères de la base de données MNIST 
grâce à  un perceptron.

                Le perceptron fait partie des architectures de réseaux neuronaux les plus simples. Son 
étude nous a donc permis de s'introduire à  la problématique du machine learning avant 
d'approfondir en étudiant 
des structures plus complexes.


            \section{La brique de base : le neurone}

                Le neurone est le composant élémentaire des réseaux neuronaux. Il est une modélisation 
du fonctionnement des neurones du systême nerveux humains.

                Chaque neurone reçoit un signal via une entrée, qui correspond aux dendrites des 
systèmes biologiques. Le neurone prend en compte la valeur de toutes ses entrées et en 
déduit la valeur de sortie. 
Cette sortie est ensuite propagée par le biais d'un axone vers un autre neurone.\\

                \begin{figure}
                    \begin{center}
                        \includegraphics[width=200pt]{Images/neurone-01.png}
                    \end{center}
                    \caption{Analogie entre neurone biologique et neurone formel.}
                \end{figure}

                La sortie du j\textsuperscript{ième} neurone est donnée par la formule :

                \begin{equation}\label{eqNeur}s_{j}(x)=f(\sum_{k=1}^{N} 
w_{j,k}*x_{k}+w_{0})\end{equation}

                où:

                \begin{itemize}
                    \item $s$ est la valeur de la sortie
                    \item $f$ est la fonction d'activation.
                    \item N est la dimension du vecteur d'entrée.
                    \item $w_{j,k}$ est la k\textsuperscript{ième} composante du vecteur de poids 
$w_{j}$ du j\textsuperscript{ième} neurone. $w_{0}$ est le biais du neurone. Cette 
valeur correspond au poids d'une 
entrée fictive valant toujours 1.
                    \item $x_{k}$ est la k\textsuperscript{ième} composante du vecteur d'entrée $x$.\\
                \end{itemize}

                L'entrée $x$ est un vecteur défini par un ensemble de caractéristiques que l'on 
choisit pour représenter les données d'entrées. Dans le cas d'une image par exemple, 
ce vecteur d'entrée peut être 
composé de la valeur de tous les pixels de l'image. D'autres représentations moins triviales peuvent 
être choisies afin de réduire la quantité d'information à  traiter et de s'approcher au mieux des 
données signifiactives de la problématique traitée.\\

                La fonction d'activation permet de définir le comportement du neurone. Selon la 
définition de cette fonction on aura une sortie à  valeurs discrètes ou continues, 
centrées en 0 ou en 0,5. Le choix de 
la fonction est donc étroitement lié avec le problème à  traiter.
                Il existe différents types de fonctions d'activation. Parmis les plus utilisées 
figurent:
                \begin{itemize}
                    \item La fonction échelon
                    \begin{equation}
                        f(x)=\mathbf{1}_{\mathbb{R}^{*}_{+}}
                    \end{equation}
                    \item Les fonctions linéaires
                    \begin{equation}
                        f(x)=\alpha*x+\beta
                    \end{equation}
                    \item La fonction sigmoïde
                    \begin{equation}
                        f(x)=\frac{1}{1+e^{-\lambda*x}}
                    \end{equation}
                    \item La fonction tangente hyperbolique
                    \begin{equation}
                        f(x)=\frac{e^{x}-e^{-x}}{e^{x}+e^{-x}}
                    \end{equation}
                \end{itemize}

                Le vecteur de poids du j\textsuperscript{ième} neurone représente la pondération de 
chacune des entrées de ce neurone. C'est ce paramètre qui permet de modifier le réseau 
de neurone sans avoir à  
modifier son architecture. Toutes les propriétés d'un réseau neuronal sont donc issues de ce vecteur 
de pondération et de ses variations.


            \section{Organisation des couches}

                Il existe de nombreuses architectures de réseau de neurone. Elles peuvent être 
récurrentes ou non, entièrement connectées ou seulement partiellement, organisées en 
couches, ... De nombreuses 
propriétés permettent de caractériser une architecture de réseau de neurone.

                Le Perceptron est un modèle assez élémentaire de réseau. Il est constitué en couches 
totalement connectées.

                \begin{figure}
                    \begin{center}
                        \includegraphics[width=200pt]{Images/perceptron-01.png}
                    \end{center}
                    \caption{Architecture d'un perceptron multi-couches.}
                \end{figure}

                On peut distinguer trois types de couches : la couche d'entrée, les couches cachées et 
la couche de sortie.\\

                La couche d'entrée est assez élémentaire. C'est tout simplement une couche dont la 
valeur sera le vecteur d'entrée $x$. Cette couche doit donc être constituée d'autant 
de neurones que le vecteur 
d'entrée a de dimensions. Les neurones de cette couche ont pour fonction d'activation l'identité.\\

                Les couches cachées sont celles qui effectuent les calculs. Les principales propriétés 
du réseau sont héritées de cet empilement de couches. On comprend alors mieux 
l'intérêt actuel pour le Deep 
Learning, c'est-à -dire l'apprentissage des réseaux avec un grand nombre de couches cachées.
                Le nombre de neurones dans chaque couche et les fonctions d'activation utilisées par 
les neurones sont choisies par le concepteur du réseau selon le problême traité par le 
réseau.\\

                Afin d'expliciter l'importance de l'architecture du réseau, abordons l'exemple usuel 
du ou exclusif.

                Essayons de réaliser avec un unique neurone la fonction logique XOR. Ce neurone aura 
deux entrées $x_{1}$ et $x_{2}$ à  valeurs dans \{0;1\} et une sortie $s$, elle aussi 
à  valeurs dans \{0;1\}. On 
prendra comme fonction d'activation la fonction de Heaviside, c'est à  dire 
$\mathbf{1}_{\mathbb{R}^{*}_{+}}$. Ce neurone aura par conséquent un fonctionnement totalement 
binaire. Il s'agit donc ici 
de déterminer les pondérations $w_{1}$ et $w_{2}$, respectivement associées aux entrées $x_{1}$ et 
$x_{2}$, qui conviennent pour obtenir en sortie la valeur $x_{1}\oplus x_{2}$.\\

                \begin{figure}
                    \begin{center}
                        \includegraphics[width=100px]{Images/xor-01.png}
                    \end{center}
                    \caption{Fonction XOR à réaliser par le perceptron.}
                \end{figure}

                Pour un tel problème, le neurone agit comme un séparateur linéaire de l'espace des 
entrées. L'équation de la droite séparant le demi-espace définit par $s=0$ de celui 
définit par $s=1$, découle alors 
directement de l'équation \ref{eqNeur}:

                \begin{equation}w_{2}*x_{2}+w_{1}*x_{1}+w_{0}=0\end{equation}

                On constate ici l'intérêt du biais, qui permet de réaliser des séparations affines de 
l'espace des entrées et pas seulement des fonctions linéaires.

                \begin{figure}
                    \begin{center}
                        \includegraphics[width=100px]{Images/biais-01.png}
                    \end{center}
                    \caption{Intérêt de l'introduction d'un biais.}
                \end{figure}

                Les limites d'un neurone seul sont alors évidentes : toutes les séparations de 
l'espace des entrées ne sont pas affine, et le cas du XOR en est déjà  une 
illustration.\\

                Il est alors nécessaire d'introduire la structure de réseau. En prennant une couche 
d'entrée, une couche cachée et une couche de sortie, respectivement composées de deux, 
deux et un neurone, on peut 
contourner le problème sus-cité. Cette structure permet en fait de générer deux séparations de 
l'espace des entrées.

                La séparation que l'on cherche à  effectuer est en effet de la forme:

                \begin{equation}
                    f(x_{1},x_{2})=0 \Leftrightarrow
                    \left\{
                        \begin{array}{r c l}
                        x_{1}+x_{2}-0,5 &<& 0\\
                        x_{1}+x_{2}-1,5 &>& 0
                        \end{array}
                    \right.
                \end{equation}\\

                Ce premier exemple simpliste manifeste donc la nécessité d'adapter la structure du 
réseau au problème traité.\\

                De manière plus générale, le nombres de neurones du réseau fait varier le nombre de 
poids du réseau, et donc la capacité du réseau à  séparer des ensembles multiples et 
complexes. Ainsi, un réseau 
sous-dimensionné mêne à  des résultats pas assez précis et un réseau sur-dimensionné mène à  du 
sur-apprentissage. On a donc environ la loi suivante:
                \begin{equation}
                    N<\frac{1}{10}*T*dim(s)
                \end{equation}
                \begin{itemize}
                    \item $N$ est le nombre de neurones dans le réseau.
                    \item $T$ est le nombre de vecteurs de la base d'apprentissage.
                    \item $s$ est le vecteur de sortie.
                \end{itemize}

                De plus, le nombre de neurones dans une couche doit être de moins de trois fois celui 
de la couche précédente.\\

                Ces deux approximations donnent donc une première idée de la structure à  adopter pour 
un réseau de neurones.



        \chapter{Algorithmes d'apprentissage}

            \section{L'objectif des algorithmes d'apprentissage}

                L'objectif des algorithmes d'apprentissage est simple : optimiser les poids du réseau 
de neurones afin qu'il "apprenne".

                Les poids sont en effet la seule variable disponible après avoir définit 
l'architecture du réseau (il existe des algorithmes permettant de modifier 
l'architecture durant l'apprentissage mais on ne les 
étudiera pas ici). On va donc les modifier progressivement, après avoir présenté au réseau un nombre 
fixé d'exemples, afin que le réseau réagisse au mieux au stimulations qui lui sont présentées.\\

                Il existe en réalité deux types d'algorithmes d'apprentissage.\\

                Le premier cas est celui de l'apprentissage supervisé. Il nécessite de connaître 
"l'étiquette" des exemples entrés dans le réseau pour effectuer l'apprentissage.

                On observe ainsi quelle est la réponse proposée par le réseau après propagation de 
l'entrée et on la compare à l'étiquette, correspondant à la réponse attendue. On peut 
alors "récompenser" les 
comportements positifs en renforçant les connections mises en jeu lors de cette propagation, ou 
"punir" les mauvais comportements en affaiblissant ces dernières.

                On s'appuie alors sur une classification a priori des données : les classes sont crées 
par le concepteur du problème et du réseau. Cela est à double tranchant. En effet, les 
classes choisies seront a 
priori plus pertinente vis-à-vis du résultat attendu  mais pas nécessairement vis-à-vis des données et 
de la structure du réseau.\\

                Le second cas est celui de l'apprentissage non supervisé. Comme l'indique son nom, ce 
genre d'apprentissage n'a pas besoin d'une intervention extérieure pour apprendre. En 
pratique cela se manifeste 
par l'absence d'étiquettes dans les bases d'apprentissage.

                Ce genre d'apprentissage revient plus ou moins à mettre en concurrence les différentes 
classes de données à chaque présentation d'exemple. Ainsi, la classe qui correspond la 
mieux à l'entrée devient 
la sortie et la classe est modifiée pour ressembler encore plus à l'entrée qui a permis l'activation.

                Au final, dans ce cas, la définition de chaque classe n'est pas explicite : c'est le 
réseau qui détecte lui même une structure de classe et il l'amplifie durant 
l'apprentissage afin d'améliorer la 
classification qu'il a détecté. Les classes ne sont alors pas forcément celles que l'on attend mais 
plutôt les classes les plus "naturelles" étant données la base d'apprentissage et l'architecture du 
réseau. En ce sens les algorithmes d'apprentissage s'approchent de la notion de "clustering".


            \section{La méthode de rétropropagation du gradient}

                La rétropropagation du gradient correspond à une descente de gradient dans l'espace 
des poids.\\

                Le principe de l'algorithme est en fait d'assigner à chaque neurone sa responsabilité 
dans l'erreur commise dans le calcul de la sortie. Les poids individuels des neurones 
sont ainsi corrigés en 
fonction de cette responsabilité.


                \subsection{Initialisation des poids}

                    Les valeurs sont initailisées au départ de manière aléatoire,avec des valeurs 
centrées en 0.

                    Le choix de l'intervalle possible est très important : un intervalle trop petit va 
beaucoup ralentir l'apprentissage tandis qu'un intervalle trop grand donne des 
données trop éparses qui rendent 
plus difficile l'apprentissage.

                    Ainsi, l'intervalle choisi est les suivant:

                    \begin{equation}
                        \forall i, w_{i}\in [\frac{-2.4}{N_{i}}, \frac{-2.4}{N_{i}}]
                    \end{equation}

                    où $N_{i}$ est le nombre d'entrées du neurone.


                \subsection{Calcul des variations de poids}

                    Les exemples sont présentés consécutivement au perceptron, et à chaque nouvel 
exemple on calcul les nouvelles variations de poids.\\

                    Une fois l'enrée propagée dans le réseau, on obtient une sortie $y^{obtenue}$. 
Notons par ailleurs la sortie théorique $y^{theorique}$ et $h$ le produit scalaire 
entre les poids $w$ et les entrées 
$x$. Nous pouvons alors calculer l'erreur commise par le réseau, pour chaque neurone de sortie :

                    \begin{equation}
                        \forall i, e_{i}^{sortie} = 
f'(h_{i}^{sortie})*(y_{i}^{theorique}-y_{i}^{obtenue})
                    \end{equation}

                    On propage ensuite l'erreur de la couche $n$ vers la couche $n-1$ en assignant à 
chaque neurone sa responsabilité dans l'erreur commise $e^{sortie}$:

                    \begin{equation}
                        \forall j, e^{(n-1)}_{j} = f'(h_{j}^{(n-1)})*\sum_{k} w_{i,j}*e_{i}^{(n)}
                    \end{equation}

                    Une fois le calcul de l'erreur retropropagé pour toutes les couches du réseau, il 
ne reste plus qu'à assigner les différentes variations de poids selon un taux 
d'apprentissage $\lambda$:

                    \begin{equation}
                        \Delta w_{i,j}^{(n)} = \lambda e_{i}^{(n)}x_{j}^{(n-1)}
                    \end{equation}


                \subsection{Potentielle modification de l'algorithme}

                    L'une des modifications possibles de cet algorithme est l'ajout d'une inertie à la 
modification des poids.\\

                    La variation de poids devient alors:

                    \begin{equation}
                        \Delta w_{i,j}^{(n)}(t) = \lambda e_{i}^{(n)}x_{j}^{(n-1)} + \alpha \Delta 
w_{i,j}^{(n)}(t-1)
                    \end{equation}

                    Le coefficient $\alpha$ est le coefficient d'inertie. Il est dans l'intervalle 
[0;1] et ajoute ainsi une influence des vrariations de poids précédentes sur les 
variations de poids actuelles.\\

                    Cette technique permet d'éviter certains minima locaux et de se stabiliser dans 
des configurations plus optimales. Cela peut aussi accélerer la convergence de 
l'apprentissage si le coefficient 
$\alpha$ est bien choisit.


        \chapter{Influence des paramètres sur l'apprentissage.}




    \part[Machines de Boltzmann]{Approfondir le sujet : les machines de Boltzmann.}

        \chapter{Introduction}

            Les réseaux de Boltzmann sont des réseaux qui ont un fonctionnement
            particulier que nous allons présenter dans ce chapitre. Les machines de
            Boltzmann constituent des réseaux neuronaux. La prinipale différence entre
            les réseaux neuronnaux comme les réseaux neuronnaux de convolution (CNN) et
            les  réseaux de Boltzmann va être le fait qu'un réseau de Boltzmann n'est
            pas \textit{feed-forward} ; comme l'était le perceptron, précédemment présenté dans ce
            rapport.

            Ce rapport traitera d'abord des machines de Boltzmann restreintes en les
            présentant et en expliquant les lois qui régissent leur fonctionnement pour
            ensuite s'intéresser aux Réseaux de Boltzmann Profond.

        \chapter{Les machines de Boltzmann restreintes}

            Les machines de Boltzmann restreintes, ou RBM, ont d'abord été
            pensées par P. Smolensky (1986) mais réellement mise en oeuvres et
            étudiées par G.
            Hinton(2006). Nous allors présenter dans cette partie leur constitution et
            leurs fonctionnement.

            \section{Éléments de preuve et présentation}

                Les RBM sont des réseaux neuronaux qui donnent une
                probabilité d'activation de chaque neurone.\\

                Une RBM possède deux couches de neurones : l'une sera qualifiée 
                de couche visible et l'autre de couche cachée.
                La couche visible correspond à l'entrée de notre réseau, ce sont
                les neurones de la couche d'entrée. On va leur assigner les
                valeurs des exemples de notre jeu de données pour entrainer le
                réseau.

                Au sein d'une même couche, aucun neurone n'est relié à un
                autre, c'est ce caractère restreint qui donne sa particularité 
                et son nom à ce type de réseau de Boltzmann.
                Cependant, chaque neurone de la couche visible (respectivement
                cachée) est connecté à tous les neurones de la couche cachée 
                (respectivement visible).\\
                
                On forme alors un graphe, non orienté, qui est un
                modèle de la distribution de probabilité. Chaque neurone est une variable aléatoire
                qui prend sa valeur dans \begin{math}{0;1}\end{math}. Le fait
                qu'aucun neurone \textit{v} ne soit connecté aux autres de la
                couche traduit le fait que les variables aléatoires associées
                sont indépendantes.
                
                La probabilité qu'un neurone \begin{math}v_{i}\end{math} de la
                couche visible soit activé est \begin{equation}p( v_{i} =1 \mid H) = 
\sigma(\sum_{j=1}^{m} w_{i,j} * h_{j} + b_{i}) \end{equation}

				En notant \textit{$w_{i}$} le poids de la synapse reliant les neurones \textit{$v_{i}$} 
à \textit{$h_{j}$} et \textit{$b_{i}$} le biais du neurones \textit{$v_{i}$}.\\
        
					Puisque la probabilité  d'activation du neurone \textit{$v_{i}$} d'une couche dépend 
uniquement de la valeur des états des neurones de l'autre couche et uniquement au 
moment où l'on souhaite 
échantillonner, l'ensemble des probabilité d'activation des neurones aux instants \textit{t} forme 
un champ de Markov en prenant pour variable aléatoire de Markov la valeur de l'état des neurones.\\

	     On introduit maintenant l'échantillonnage de Gibbs, qui est un
                algorithme de la classe des Metropolis-Hasting. Cet algorithme
                est un Monte-Carlo qui permet d'échantillonner un probabilité jointe dont l'idée principale
                est d'échantillonner les états des variables en se basant sur la valeur des états
                des autres variables grâce à la probabilité conditionnelle qui les relie. C'est une marche aléatoire sur les chaînes de Markov.
Les neurones d'une même couche sont indépendants on peut donc mettre à jour une couche 
entière d'un coup ce qui rend l'échantillonnage plus rapide et plus simple à réaliser. \\

Par ailleurs on peut démontrer que tout les neurones peuvent prendre les valeur 0 ou 1
 en un seul échantillonnage et que toutes les combinaisons d'état des neurones de la RBM 
sont atteignables en un nombre fini d'échantillonnage ce qui montre que la chaîne est irréductible. Par ailleurs la la chaîne de Markov est apériodique. 
Ces deux propriété d'apériodicité et de irréductibilité permettent de prouver que le RBM
 possède une distribution stationnaire vers laquelle elle converge. 

Une preuve plus rigoureuse à été proposé par Fischer et Igel (\textit{An Introduction to Restricted Boltzmann Machines} (2012).\\

 La particularité de ce réseau est donc le fait qu'il n'est pas
                \textit{feed-forward}, même si l'on considère la couche cachée comme entrée du
                réseau, le réseau arrive à une distribution stationnaire qu'après des allers-retours (c'est-à-dire des échantillonnage) entre
les deux couches.

\section{Loi d'apprentissage}

Pour faire l'apprentissage, on va entraîner le réseau sur les différents exemple, la première différence avec l'implémentation du perceptron est
que l'on va  échantillonner les états des neurones à chaque fois, on travaille avec des données binaires et non réelles, nous avons décidé 
d'affecter, dans le cas de la base de données MNIST, la valeur 1 si le pixel n'était pas blanc.
On va ensuite faire des pas de Gibbs pour obtenir la distribution stationnaire de probabilité
 qui correspond à l'exemple et aux paramètres (poids et biais) fixés.
\begin{figure}[!h]
                    \begin{center}
                        \includegraphics[width=200pt]{Images/Gibbs.png}
                    \end{center}
                    \caption{Pas de Gibbs}
\end{figure}\\

Une fois la distribution stationnaire atteintes on obtient deux ensembles de valeurs : l'état des couches \textit{V} et  \textit{H}
qui correspond à l'\textit{input}, que l'on peut appeler \begin{math}<v_{i}h_{j}>_{data}\end{math} ou \begin{math}<v_{i}^{0}h_{j}^{0}>\end{math}, et l'état des couches \textit{V} et  \textit{H} une fois la distribution stationnaire atteinte notée \begin{math}<v_{i}h_{j}>_{model}\end{math} ou \begin{math}<v_{i}^{\infty}h_{j}^{\infty}>\end{math}. A partir de ces deux distribution, on peut modifier les poids et les biais du réseau selon la loi :
                \begin{equation}
                    \Delta w_{i,j} = \alpha (<v_{i}h_{j}>_{data} - <v_{i}h_{j}>_{model} )
                \end{equation}
                où \begin{math}\alpha\end{math} est le \textit{learning rate}\\

                Un point critique de l'implémentation d'une RBM est comme bien souvent la loi 
d'apprentissage.
                Si l'on trouve dans la littérature plusieurs solutions, la forme générale du calcul 
est celle présentée ci-dessus.\\

                Dans ce calcul, on distingue l'échantillonnage de la machine à partir des données 
(data), de l'échantillonnage de la machine après la reconstruction par pas de Gibbs, 
\begin{math}<v_{i}h_{j}>_{model}\end{math}.\\
                
                Pour extraire des informations à partir de l'\textit{input}, il est important de ne 
pas reconstruire les deux couches depuis les \textit{inputs}.
                Ainsi le calcul du produit \begin{math}<v_{i}h_{j}>_{data}\end{math} s'obtient par 
l'état de l'entité visible \begin{math}v_{i}\end{math} et l'évaluation de l'entité 
cachée sous forme réelle 
: \begin{math}p(h_{j}|v^{(0)})\end{math}.\\

Une chaîne de Markov peut prendre du temps à converger et dans le cas des RBM qui peuvent 
avoir plusieurs centaines de neurones par couche, les pas de Gibbs, c'est-à-dire l'échantillonnage
 des états des neurones au sein d'une même couche, peut être très calculatoire. Il est alors proposé par 
Hinton en 2006 une méthode pour approximer la distribution stationnaire : au lieu de faire un nombre
 fini de pas de Gibbs jusqu'à avoir cette stationnarité, on se limite qu'à peu de pas de Gibbs. Cette 
méthode s'est révélée efficace lors de l'apprentissage des RBM. On se limite à quelques pas de Gibbs (de l'ordre de l'unité).
Hinton propose d'ailleurs de ne faire qu'un seul pas de Gibbs.







                				

        \chapter{Les Deep Belief Networks}



        \chapter{Aller plus loin}

            \section{La visualisation des spectres du réseau}

\subsection{Cas de la RBM}

                Lors du développement de la RBM, nous avons étés confrontés au problème de la 
quantification de l'apprentissage. Comment savoir si oui ou non le réseau s'adapte aux 
exemples qu'on lui a présenté ?\\

                Suite à l'absence de réelle mesure de cette notion, l'idée nous est venue 
d'implémenter une mesure bien plus qualitative du phénomène : l'objectif était de 
visualiser l'importance que le réseau 
accorde à chaque pixel de l'image.\\

                Dans le cas de la RBM, cela correspond tout simplement à l'ensemble des pondérations 
des connections entre la couche visible et la couche cachée du réseau. Ainsi, à chaque 
neurone de la couche cachée, 
on peut associer une image correspondant aux dites pondérations. Ce sont ces images que l'ont qualifie 
de "spectres".

                \begin{figure}
                    \begin{center}
                        \includegraphics[width=150pt]{Images/filters-01.png}
                    \end{center}
                    \caption{Spectre d'une RBM entraînée à vingt neurones cachés.}
                \end{figure}

                Ces spectres représentent les caractéristiques que le réseau essaie de détecter dans 
une image pour y reconnaître un caractère. Selon le nombre de neurones dans la couche 
cachée, ces caractéristiques 
sont assez différentes. En effet, plus il y a de neurones cachés, plus le réseau pourra détecter de 
nombreuses caractéristiques précises, qui une fois assemblées formeront la représentation du 
caractère complet. Ces caractéristiques peuvent être, par exemple, des angles ou des boucles. A 
contrario, quand le réseau a peu de neurones cachés, ces spectres sont beaucoup plus proches de la 
forme 
"usuelle" du caractère associé. Ainsi, avec un seul neurone caché, on obtient très exactement la 
représentation d'un chiffre idéal pour ce réseau.\\

                L'intérêt d'une telle initiative est simple : utiliser l'intuition visuelle qu'on a 
des chiffres pour juger de la qualité de l'apprentissage.

                \begin{figure}
                    \begin{center}
                        \includegraphics[width=75pt]{Images/spectres-01.png}
                    \end{center}
                    \caption{Spectre d'une RBM ayant apprise la "forme" 4.}
                \end{figure}

                Au début de l'apprentissage, l'image sera exclusivement composée de bruit blanc (c'est 
une conséquence de l'algorithme d'apprentissage), tandis qu'à la fin elle ressemblera 
au négatif d'un des 
exemples de la base de donnée (toujours dans le cas d'un réseau à un seul neurone caché).\\

Pour faire cette visualisation, on cherche uniquement à connaître les neurones qui activent le neurone de la couche caché. 
Une méthode de visualisation serait donc de chercher les valeurs binaires  des neurones de la couche visible pour 
activer le neurone désiré. Puisque le réseau n'est pas orienté, on peut directement affecter la valeur 1 au neurone \textit{$h_{j}$} 
de la couche caché qui correspond au filtre que l'on veut voir et 0 aux autres. On fait alors un pas de Gibbs vers la couche visible. Pour obtenir un meilleur résultat, 
on peut directement travailler avec la valeur de la sigmoïde des neurones de la couche visible.

\subsection{Cas du DBN}

Dans le cas du DBN, il est plus compliqué d'obtenir une image du fait de l'architecture profonde. On va donc choisir le neurone de la couche dont  on veut observer le filtre ou ce qu'on pourrait appeler la carte d'activation. Pour cela on assigne la valeur 0 à tout les neurones du réseau et on va faire un unique pas de Gibbs jusqu'à la couche d'entrée du réseau en échantillonnant la valeur des neurones des couches par lesquelles on passe.\\

Cette méthode a donc un inconvénient : on a une chance de perdre de l'information. Nous avons alors testé deux options : la première est de changer la nature du réseau et de travailler avec les probabilités comme valeur de l'état du réseau. Cette méthode n'a pas été concluante. Nous sommes alors venus à considérer une autre méthode, dans l'esprit du réseau. \\

Cette méthode est la suivante : on initialise les valeurs des neurones du réseau entraîné à 0 sauf pour le filtre que l'on souhaite visualiser qui est à l'état 1. On va ensuite effectuer plusieurs pas de Gibbs au sein de la \textbf{même} RBM. Ainsi on va obtenir une distribution des états qui correspond à l'activation du filtre désiré mais aussi d'autres filtres. Ce procédé permet de réduire le nombre de neurones de la couche précédente qui ne serait pas dan l'état voulu. On procède ainsi jusqu'à la première RBM où l'on utilise la méthode exposée dans la partie correspondante.\\

Cette méthode est simplement expérimentale mais produit des résultats plus probant. Dans la littérature les méthodes de visualisation sont peu détaillées mais semble être le résultat d'un apprentissage à partir d'une entrée quelconque afin d'avoir le neurone dont on veut observer la carte d'activation activé. C'est à dire que l'on ferait un apprentissage supervisé mais on ne modifierait que la valeur d'entrée du réseau.




    \appendix

        \chapter{La base MNIST}

            \section{Présentation de la base de donées}

                La base de données MNIST, Mixed Mixed National Institute of Standards and Technology, 
est un ensemble de chiffres manuscrits extraits d'une base antérieure, NIST. Lea base 
est organisée de la manière 
suivante:

                \begin{itemize}
                    \item La base d'apprentissage est constituée de 60000 images.
                    \item La base de test est conctituée de 10000 images.
                    \item Chaque image est constituée de 28*28 pixels en nuances de gris.
                    \item Chaque image est normalisée.\\
                \end{itemize}

                Cette base est aujourd'hui devenue une référence permettant de tester les algorithmes 
d'apprentissages ou de reconnaissance de forme divers et variés. Cet aspect facilite 
ainsi la comparaison de nos 
résultats avec ceux d'autres personnes ayant fait la même démarche que nous.\\

                On peut trouver cette base à l'adresse : \url{http://yann.lecun.com/exdb/mnist/}

                \begin{figure}
                    \begin{center}
                        \includegraphics{Images/mnist-01.png}
                    \end{center}
                    \caption{Exemples de chiffres de la base de donnée MNIST.}
                \end{figure}


            \section{Limites de cette base}

                Cette base, malgré tous ses avantages, n'est cependant pas parfaite.\\

                En effet, une première limite apparaît assez rapidement quand on travaille avec MNIST 
: cette base n'est pas "universelle". Les caractères utilisés dans la base de donnée 
sont fortement connotés 
anglosaxons. On peut constater cet aspect dans les spectres réalisés par nos RBM.

                L'image idéale de chaque chiffre a toujours une prédominante américaine, et quand il 
s'agît ensuite d'entrer des données européenne, la limitation de la base se révèle.\\

                \begin{figure}
                    \centering
                    \begin{subfigure}[b]{0.3\textwidth}
                        \centering
                        \includegraphics{Images/mnist-02.png}
                        \caption{"1" en français}
                    \end{subfigure}
                    \hfill
                    \begin{subfigure}[b]{0.3\textwidth}
                        \centering
                        \includegraphics{Images/mnist-03.png}
                        \caption{"1" anglosaxon}
                    \end{subfigure}
                    \caption{Exemple d'une situation de litige entre notation anglosaxone et française.}
                \end{figure}

                Une autre limite apparaît aussi : c'est celle de la simplicité de la base de donnée.

                La reconnaissance de caractères est en effet une tâche simpliste en comparaison avec 
certains des problèmes auxquels peuvent être confrontés les réseaux issus du deep 
learning. Cela se constate par 
les résultats très bons obtenus pour des réseaux peu complexes, tel que notre perceptron qui 
atteignait les 3\% d'erreur sur MNIST.

                Il devient alors nécessaire de trouver un test plus discriminatoire de la qualité des 
réseaux.\\

                Faute de temps nous n'avons pas eu la possiblité d'approfondir ce point de vue mais 
semble nécessaire pour tester les réseaux dans un domaine plus complexe.

        \chapter{Présentation du code}


\end{document}
